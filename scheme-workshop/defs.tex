\newcommand{\warn}[1]{\PackageWarning{dusa}{#1}}
\newcommand{\depr}{\textcolor{Red}{\textnormal{DEPRECATED}}\warn{DEPRECATED}}
\NewDocumentCommand{\authnote}{smm}{\IfBooleanTF{#1}{\footnote}{\identity}{\textcolor{Red}{\textnormal{#2: #3}}}}
\newcommand{\makeauthnote}[1]{%
  \expandafter\NewDocumentCommand\csname #1\endcsname{sm}{%
    \IfBooleanTF{##1}{\footnote}{\identity}{\textcolor{Red}{\textnormal{\MakeUppercase{#1}: ##2}}}%
  }%
}%
\makeauthnote{zje}
\let\textlabel\label
\newcommand{\identity}[1]{#1}

%%%%%%%%%%%%%%%%%%%%%%%%%%%%%%%%%%%%%%%%%%%%%%%%%%%%%%%%%
%%%%%%%%%%%%%%%%%%%% TEXT FORMATTING %%%%%%%%%%%%%%%%%%%%
%%%%%%%%%%%%%%%%%%%%%%%%%%%%%%%%%%%%%%%%%%%%%%%%%%%%%%%%%

\Crefname{section}{\S}{\S\S}
\Crefname{figure}{Fig.}{Figs.}

\newcommand{\kw}[1]{\mathcolor{MidnightBlue}{\mathsf{#1}}}
\newcommand{\Kw}[1]{\mathsf{#1}}

\makeatletter
\newcommand{\labitem}[1]{\item[#1]\def\@currentlabelname{#1}}
\makeatother
\newcommand{\placehold}[1][0.13]{\includegraphics[width=\textwidth,height=#1\textheight]{example-image}}

% name [label] default [style]
\NewDocumentCommand{\makestx}{smomO{\identity}O{#5}}{%
  % starred version lets you change the base symbol, useful for variables and numbers
  \IfBooleanTF{#1}{%
    \expandafter\NewDocumentCommand\csname #2\endcsname{O{#4} e{_^}}{%
      \IfValueTF{#3}{\newlink{#3}}{\identity}{%
        #5{##1}\IfValueT{##2}{_{#6{\mathsmaller{##2}}}}\IfValueT{##3}{^{#6{\mathsmaller{##3}}}}%
      }%
    }%
  }
  % non-starred version has a fixed base symbol, useful for expr / etc
  {%
    \expandafter\NewDocumentCommand\csname #2\endcsname{e{_^}}{%
      \IfValueTF{#3}{\newlink{#3}}{\identity}{%
        #5{#4}\IfValueT{##1}{_{#6{\mathsmaller{##1}}}}\IfValueT{##2}{^{#6{\mathsmaller{##2}}}}%
      }%
    }%
  }%
}%

%%%%%%%%%%%%%%%%%%%%%%%%%%%%%%%%%%%%%%%%%%%%%%%%%%%%%%
%%%%%%%%%%%%%%%%%%%%%%%% LANG %%%%%%%%%%%%%%%%%%%%%%%%
%%%%%%%%%%%%%%%%%%%%%%%%%%%%%%%%%%%%%%%%%%%%%%%%%%%%%%

% stub

%%%%%%%%%%%%%%%%%%%%%%%%%%%%%%%%%%%%%%%%%%%%%%%%%%%%%%
%%%%%%%%%%%%%%%%%%%%%%%% TYPES %%%%%%%%%%%%%%%%%%%%%%%
%%%%%%%%%%%%%%%%%%%%%%%%%%%%%%%%%%%%%%%%%%%%%%%%%%%%%%

% stub

%%%%%%%%%%%%%%%%%%%%%%%%%%%%%%%%%%%%%%%%%%%%%%%%%%%%%%
%%%%%%%%%%%%%%%%%%%%%%% SYMBOLS %%%%%%%%%%%%%%%%%%%%%%
%%%%%%%%%%%%%%%%%%%%%%%%%%%%%%%%%%%%%%%%%%%%%%%%%%%%%%

\newcommand{\BB}{\mathbb{B}}
\newcommand{\NN}{\mathbb{N}}
\newcommand{\PP}{\mathbb{P}}
\newcommand{\UU}{\mathbbm{1}}
\newcommand{\ZZ}{\mathbb{Z}}
\newcommand{\poison}{\text{\Biohazard}}
\newcommand{\sep}{\star}
\newcommand{\bigsep}{\scalerel*{\sep}{\sum}}

%%%%%%%%%%%%%%%%%%%%%%%%%%%%%%%%%%%%%%%%%%%%%%%%%%%%%%
%%%%%%%%%%%%%%%%%%%%%% OPERATORS %%%%%%%%%%%%%%%%%%%%%
%%%%%%%%%%%%%%%%%%%%%%%%%%%%%%%%%%%%%%%%%%%%%%%%%%%%%%

\RenewDocumentCommand{\vec}{m E{^}{{}}}{\overline{#1}^{\mathsmaller{#2}}}
\newcommand{\sem}[1]{{\left\llbracket #1 \right\rrbracket}}
\newcommand{\from}{\leftarrow}
\newcommand{\To}{\Rightarrow}
\newcommand{\nto}{\nrightarrow}
\newcommand{\defeq}{\triangleq}
\newcommand{\defapx}{\stackrel{\triangle}{\approx}}
\newcommand{\dom}{\textnormal{dom}}
\newcommand{\symdiff}{\mathrel{\triangle}}
% https://tex.stackexchange.com/a/22255
\newcommand\restr[2]{{% we make the whole thing an ordinary symbol
  \left.\kern-\nulldelimiterspace % automatically resize the bar with \right
  #1 % the function
  \vphantom{\big|} % pretend it's a little taller at normal size
  \right|_{#2} % this is the delimiter
  }}
\newcommand{\pto}{\stackrel{\scriptscriptstyle \kw{fin}}{\rightharpoonup}}
\renewcommand{\implies}{\Rightarrow}
\renewcommand{\iff}{\Leftrightarrow}
\let\oldforall\forall
\renewcommand{\forall}{\oldforall\,}
\let\oldexists\exists
\renewcommand{\exists}{\oldexists\,}

% https://damaru2.github.io/general/notations_with_links/
\newcommand\newlink[2]{{\protect\hyperlink{#1}{\normalcolor #2}}}
\makeatletter
\def\Hy@raisedlink@left#1{%
    \ifvmode
        #1%
    \else
        \Hy@SaveSpaceFactor
        \llap{\smash{% \llapadded
        \begingroup
            \let\HyperRaiseLinkLength\@tempdima
            \setlength\HyperRaiseLinkLength\HyperRaiseLinkDefault
            \HyperRaiseLinkHook
        \expandafter\endgroup
        \expandafter\raise\the\HyperRaiseLinkLength\hbox{%
            \Hy@RestoreSpaceFactor
            #1%
            \Hy@SaveSpaceFactor%
        }} } %to close \llap
        \Hy@RestoreSpaceFactor
        \penalty\@M\hskip\z@
    \fi
}
\newcommand\newtarget[2]{\Hy@raisedlink{\hypertarget{#1}{}}#2}
\makeatother

%%%%%%%%%%%%%%%%%%%%%%%%%%%%%%%%%%%%%%%%%%%%%%%%%%%%%%
%%%%%%%%%%%%%%%%%%%%%%% PROOFS %%%%%%%%%%%%%%%%%%%%%%%
%%%%%%%%%%%%%%%%%%%%%%%%%%%%%%%%%%%%%%%%%%%%%%%%%%%%%%

\theoremstyle{acmplain}
\newtheorem{theorem}{Theorem}[section]
\newtheorem{lemma}[theorem]{Lemma}
\newtheorem{convention}[theorem]{Convention}

% this is mostly a hack because I couldn't figure out a better way
\newcommand{\convref}[1]{%
  \hyperref[#1]{\textsc{Convention}~\ref{#1}}%
}
\newcommand{\figref}[1]{%
  \hyperref[#1]{Figure~\ref{#1}}%
}

\newcounter{hyp}
\newcounter{gol}
\newcounter{def}
\counterwithin*{hyp}{theorem}
\counterwithin*{gol}{theorem}
\counterwithin*{def}{theorem}
\newcommand{\labelhyp}[1]{hyp:\thetheorem:#1}
\newcommand{\labelgol}[1]{gol:\thetheorem:#1}
\newcommand{\labeldef}[1]{def:\thetheorem:#1}

\newcommand{\hyptarget}[2]{%
  \begingroup\refstepcounter{hyp}\textlabel{\expandafter\labelhyp{#1}}%
    {\newtarget{\expandafter\labelhyp{#1}}{{#2}^{(\hyplink{#1})}}}\endgroup
}
\newcommand{\hyplink}[1]{\newlink{\expandafter\labelhyp{#1}}{\text{H\ref{\expandafter\labelhyp{#1}}}}}

\NewDocumentCommand{\goltarget}{mm}{%
  \begingroup\refstepcounter{gol}\textlabel{\expandafter\labelgol{#1}}%
    {\newtarget{\expandafter\labelgol{#1}}{{#2}^{(\gollink{#1})}}}\endgroup
}
\newcommand{\gollink}[1]{\newlink{\expandafter\labelgol{#1}}{\text{G\ref{\expandafter\labelgol{#1}}}}}

\NewDocumentCommand{\deftarget}{mm}{%
  \begingroup\refstepcounter{def}\textlabel{\expandafter\labeldef{#1}}%
    {\newtarget{\expandafter\labeldef{#1}}{{#2}^{(\deflink{#1})}}}\endgroup
}
\newcommand{\deflink}[1]{\newlink{\expandafter\labeldef{#1}}{\text{G\ref{\expandafter\labeldef{#1}}}}}

\newcommand{\assume}{\hyptarget}
\newcommand{\have}{\hyplink}
\newcommand{\by}{\hyplink}
\newcommand{\wts}{\goltarget}
\newcommand{\suff}{\goltarget}
\newcommand{\shows}{\gollink}
\newcommand{\exact}{\gollink}
\newcommand{\pose}{\deftarget}

\newenvironment{customfont}[2]{%
    \begingroup%
    \fontsize{#1}{#2}\selectfont%
}{%
    \endgroup%
}

\newcommand{\mprlabpos}{lab}
% \newcommand{\mprlabpos}{right}

\let\oldinferrule\inferrule
% adapted from
% https://tex.stackexchange.com/a/340843
\makeatletter
\DeclareDocumentCommand \inferrule { s O {} O{\mprlabpos} m m o}{%
  \IfBooleanTF{#1}%
  {%
    \oldinferrule*[#2]{#4}{#5}%
  }{%
    \oldinferrule*[%
    % HACK WARNING: using #3 and not {#3} so I can add side conditions
    #3=(\IfValueTF{#6}{%
      \begingroup%
      \def\@currentlabelname{\textsc{#2}}%
      \phantomsection%
      \textlabel{#6}{\newtarget{#6}{#2}}%
      \endgroup%
    }%
    {#2}%
    )%
    ]{#4}{#5}%
  }%
}
\makeatother

% surely there is a better way to do this
% this wrapper macro is at least better than the previous hack
\DeclareDocumentCommand \inferruleside { s O {} m m m o}{%
  \IfBooleanTF{#1}{%
    \inferrule*[#2][right={(\({#3}\))}, \mprlabpos]{#4}{#5}[#6]%
  }{%
    \inferrule[#2][right={(\({#3}\))}, \mprlabpos]{#4}{#5}[#6]%
  }%
}

\newcommand{\dash}{\text{-}}
\ebproofnewstyle{back}{proof style=downwards}
\ebproofnewstyle{small}{
  separation = 1em, rule margin = .5ex,
  template = {\small$\inserttext$},
  right label template = {\small\inserttext}
}
\ebproofnewstyle{smaller}{
  separation = 1em, rule margin = .5ex,
  template = {\footnotesize$\inserttext$},
  right label template = {\footnotesize\inserttext}
}

\newenvironment{mathline}{\\[\abovedisplayskip] $}{$ \\[\belowdisplayskip]}

\newcommand{\namerefsc}[1]{\textsc{\nameref{#1}}}

\newcommand{\pinfer}{%
  % \mprset{myfraction=\entailfrac} % breaks on large cases
  \oldinferrule
}

\newcommand{\processnameref}[1]{\nameref{#1}, }
\NewDocumentCommand{\namerefs}{>{\SplitList{,}}om}{%
  \IfValueT{#1}{\ProcessList{#1}{\processnameref} and }\nameref{#2}%
}

%% modified from iris: https://gitlab.mpi-sws.org/iris/iris/-/blob/master/tex/pftools.sty?ref_type=heads
\newcommand{\entailfrac}[2]{%
  \hbox{%
    \ooalign{%
      $\mid\xjoinrel[9]\genfrac{}{}{1.6pt}1{#1}{#2}$\cr%
      $\mid\xjoinrel[9]\color{white}\genfrac{}{}{0.8pt}1{\phantom{#1}}{\phantom{#2}}$%
    }%
  }%
}

% https://tex.stackexchange.com/a/114367
\makeatletter
\providecommand*{\Dashv}{%
  \mathrel{%
    \mathpalette\@Dashv\vDash
  }%
}
\newcommand*{\@Dashv}[2]{%
  \reflectbox{$\m@th#1#2$}%
}
\makeatother

% https://tex.stackexchange.com/a/34716
\makeatletter
\providecommand*{\nest}{%
  \mathrel{%
    \mathpalette\@nest\Rsh
  }%
}
\newcommand*{\@nest}[2]{%
  \raisebox{\depth}{\scalebox{1}[-1]{$\m@th#1#2$}}%
}
\makeatother
