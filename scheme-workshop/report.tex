\documentclass[dvipsnames,sigplan,screen,review,anonymous,acmthm,nonacm]{acmart}
% \documentclass[dvipsnames,sigplan,screen,acmthm]{acmart}

%%
%% \BibTeX command to typeset BibTeX logo in the docs
\AtBeginDocument{%
  \providecommand\BibTeX{{%
    \normalfont B\kern-0.5em{\scshape i\kern-0.25em b}\kern-0.8em\TeX}}}

%% Rights management information.  This information is sent to you
%% when you complete the rights form.  These commands have SAMPLE
%% values in them; it is your responsibility as an author to replace
%% the commands and values with those provided to you when you
%% complete the rights form.
\setcopyright{none}
% \setcopyright{acmcopyright}
% \copyrightyear{2018}
% \acmYear{2018}
% \acmDOI{10.1145/1122445.1122456}

%% TODO: does this matter?

% \citestyle{acmauthoryear}
\citestyle{acmnumeric}

% https://authors.acm.org/proceedings/production-information/accepted-latex-packages

% \usepackage[left=0.5in, right=0.5in, top=1in, bottom=1in]{geometry}
\usepackage{amsmath,amsthm}
\usepackage[inline]{enumitem}
\usepackage{cleveref}
\usepackage{hyperref}
\usepackage{caption}
\usepackage{subcaption}
\usepackage{stmaryrd}
% \usepackage{xspace}
\usepackage{mathpartir}
\usepackage{scalerel}
% \usepackage{wasysym}
% \usepackage{MnSymbol}
\usepackage{bbm}
\usepackage{bm}
% https://tex.stackexchange.com/a/112585
% \usepackage{amstext} % for \text macro
% \usepackage{array}   % for \newcolumntype macro
\usepackage{marvosym}
\usepackage{stackengine}
% \usepackage{calc}
\usepackage{xifthen}
\usepackage{xparse}
\usepackage{nameref}
\usepackage[allcommands]{overarrows}
\usepackage{ebproof}
\usepackage{tikz}
\usepackage{tikz-layers}
\usepackage{relsize}
\usetikzlibrary{calc, matrix, positioning, fit, arrows.meta}
\PassOptionsToPackage{names,dvipsnames}{xcolor}
\PassOptionsToPackage{colorlinks=true}{hyperref}
\mprset {sep=1em}

\newcommand{\warn}[1]{\PackageWarning{dusa}{#1}}
\newcommand{\depr}{\textcolor{Red}{\textnormal{DEPRECATED}}\warn{DEPRECATED}}
\NewDocumentCommand{\authnote}{smm}{\IfBooleanTF{#1}{\footnote}{\identity}{\textcolor{Red}{\textnormal{#2: #3}}}}
\newcommand{\makeauthnote}[1]{%
  \expandafter\NewDocumentCommand\csname #1\endcsname{sm}{%
    \IfBooleanTF{##1}{\footnote}{\identity}{\textcolor{Red}{\textnormal{\MakeUppercase{#1}: ##2}}}%
  }%
}%
\makeauthnote{zje}
\let\textlabel\label
\newcommand{\identity}[1]{#1}

%%%%%%%%%%%%%%%%%%%%%%%%%%%%%%%%%%%%%%%%%%%%%%%%%%%%%%%%%
%%%%%%%%%%%%%%%%%%%%%%% SHORTHANDS %%%%%%%%%%%%%%%%%%%%%%
%%%%%%%%%%%%%%%%%%%%%%%%%%%%%%%%%%%%%%%%%%%%%%%%%%%%%%%%%

\newcommand{\Dusa}{Dusa}
\newcommand{\miniDusa}{miniDusa}
\newcommand{\syntaxspec}{\texttt{syntax-spec}}


%%%%%%%%%%%%%%%%%%%%%%%%%%%%%%%%%%%%%%%%%%%%%%%%%%%%%%%%%
%%%%%%%%%%%%%%%%%%%% TEXT FORMATTING %%%%%%%%%%%%%%%%%%%%
%%%%%%%%%%%%%%%%%%%%%%%%%%%%%%%%%%%%%%%%%%%%%%%%%%%%%%%%%

\Crefname{section}{\S}{\S\S}
\Crefname{figure}{Fig.}{Figs.}

\newcommand{\kw}[1]{\mathcolor{MidnightBlue}{\mathsf{#1}}}
\newcommand{\Kw}[1]{\mathsf{#1}}

\makeatletter
\newcommand{\labitem}[1]{\item[#1]\def\@currentlabelname{#1}}
\makeatother
\newcommand{\placehold}[1][0.13]{\includegraphics[width=\textwidth,height=#1\textheight]{example-image}}

% name [label] default [style]
\NewDocumentCommand{\makestx}{smomO{\identity}O{#5}}{%
  % starred version lets you change the base symbol, useful for variables and numbers
  \IfBooleanTF{#1}{%
    \expandafter\NewDocumentCommand\csname #2\endcsname{O{#4} e{_^}}{%
      \IfValueTF{#3}{\newlink{#3}}{\identity}{%
        #5{##1}\IfValueT{##2}{_{#6{\mathsmaller{##2}}}}\IfValueT{##3}{^{#6{\mathsmaller{##3}}}}%
      }%
    }%
  }
  % non-starred version has a fixed base symbol, useful for expr / etc
  {%
    \expandafter\NewDocumentCommand\csname #2\endcsname{e{_^}}{%
      \IfValueTF{#3}{\newlink{#3}}{\identity}{%
        #5{#4}\IfValueT{##1}{_{#6{\mathsmaller{##1}}}}\IfValueT{##2}{^{#6{\mathsmaller{##2}}}}%
      }%
    }%
  }%
}%

%%%%%%%%%%%%%%%%%%%%%%%%%%%%%%%%%%%%%%%%%%%%%%%%%%%%%%
%%%%%%%%%%%%%%%%%%%%%%%% LANG %%%%%%%%%%%%%%%%%%%%%%%%
%%%%%%%%%%%%%%%%%%%%%%%%%%%%%%%%%%%%%%%%%%%%%%%%%%%%%%

% stub

%%%%%%%%%%%%%%%%%%%%%%%%%%%%%%%%%%%%%%%%%%%%%%%%%%%%%%
%%%%%%%%%%%%%%%%%%%%%%%% TYPES %%%%%%%%%%%%%%%%%%%%%%%
%%%%%%%%%%%%%%%%%%%%%%%%%%%%%%%%%%%%%%%%%%%%%%%%%%%%%%

% stub

%%%%%%%%%%%%%%%%%%%%%%%%%%%%%%%%%%%%%%%%%%%%%%%%%%%%%%
%%%%%%%%%%%%%%%%%%%%%%% SYMBOLS %%%%%%%%%%%%%%%%%%%%%%
%%%%%%%%%%%%%%%%%%%%%%%%%%%%%%%%%%%%%%%%%%%%%%%%%%%%%%

\newcommand{\BB}{\mathbb{B}}
\newcommand{\NN}{\mathbb{N}}
\newcommand{\PP}{\mathbb{P}}
\newcommand{\UU}{\mathbbm{1}}
\newcommand{\ZZ}{\mathbb{Z}}
\newcommand{\poison}{\text{\Biohazard}}
\newcommand{\sep}{\star}
\newcommand{\bigsep}{\scalerel*{\sep}{\sum}}

%%%%%%%%%%%%%%%%%%%%%%%%%%%%%%%%%%%%%%%%%%%%%%%%%%%%%%
%%%%%%%%%%%%%%%%%%%%%% OPERATORS %%%%%%%%%%%%%%%%%%%%%
%%%%%%%%%%%%%%%%%%%%%%%%%%%%%%%%%%%%%%%%%%%%%%%%%%%%%%

\RenewDocumentCommand{\vec}{m E{^}{{}}}{\overline{#1}^{\mathsmaller{#2}}}
\newcommand{\sem}[1]{{\left\llbracket #1 \right\rrbracket}}
\newcommand{\from}{\leftarrow}
\newcommand{\To}{\Rightarrow}
\newcommand{\nto}{\nrightarrow}
\newcommand{\defeq}{\triangleq}
\newcommand{\defapx}{\stackrel{\triangle}{\approx}}
\newcommand{\dom}{\textnormal{dom}}
\newcommand{\symdiff}{\mathrel{\triangle}}
% https://tex.stackexchange.com/a/22255
\newcommand\restr[2]{{% we make the whole thing an ordinary symbol
  \left.\kern-\nulldelimiterspace % automatically resize the bar with \right
  #1 % the function
  \vphantom{\big|} % pretend it's a little taller at normal size
  \right|_{#2} % this is the delimiter
  }}
\newcommand{\pto}{\stackrel{\scriptscriptstyle \kw{fin}}{\rightharpoonup}}
\renewcommand{\implies}{\Rightarrow}
\renewcommand{\iff}{\Leftrightarrow}
\let\oldforall\forall
\renewcommand{\forall}{\oldforall\,}
\let\oldexists\exists
\renewcommand{\exists}{\oldexists\,}

% https://damaru2.github.io/general/notations_with_links/
\newcommand\newlink[2]{{\protect\hyperlink{#1}{\normalcolor #2}}}
\makeatletter
\def\Hy@raisedlink@left#1{%
    \ifvmode
        #1%
    \else
        \Hy@SaveSpaceFactor
        \llap{\smash{% \llapadded
        \begingroup
            \let\HyperRaiseLinkLength\@tempdima
            \setlength\HyperRaiseLinkLength\HyperRaiseLinkDefault
            \HyperRaiseLinkHook
        \expandafter\endgroup
        \expandafter\raise\the\HyperRaiseLinkLength\hbox{%
            \Hy@RestoreSpaceFactor
            #1%
            \Hy@SaveSpaceFactor%
        }} } %to close \llap
        \Hy@RestoreSpaceFactor
        \penalty\@M\hskip\z@
    \fi
}
\newcommand\newtarget[2]{\Hy@raisedlink{\hypertarget{#1}{}}#2}
\makeatother

%%%%%%%%%%%%%%%%%%%%%%%%%%%%%%%%%%%%%%%%%%%%%%%%%%%%%%
%%%%%%%%%%%%%%%%%%%%%%% PROOFS %%%%%%%%%%%%%%%%%%%%%%%
%%%%%%%%%%%%%%%%%%%%%%%%%%%%%%%%%%%%%%%%%%%%%%%%%%%%%%

\theoremstyle{acmplain}
\newtheorem{theorem}{Theorem}[section]
\newtheorem{lemma}[theorem]{Lemma}
\newtheorem{convention}[theorem]{Convention}

% this is mostly a hack because I couldn't figure out a better way
\newcommand{\convref}[1]{%
  \hyperref[#1]{\textsc{Convention}~\ref{#1}}%
}
\newcommand{\figref}[1]{%
  \hyperref[#1]{Figure~\ref{#1}}%
}

\newcounter{hyp}
\newcounter{gol}
\newcounter{def}
\counterwithin*{hyp}{theorem}
\counterwithin*{gol}{theorem}
\counterwithin*{def}{theorem}
\newcommand{\labelhyp}[1]{hyp:\thetheorem:#1}
\newcommand{\labelgol}[1]{gol:\thetheorem:#1}
\newcommand{\labeldef}[1]{def:\thetheorem:#1}

\newcommand{\hyptarget}[2]{%
  \begingroup\refstepcounter{hyp}\textlabel{\expandafter\labelhyp{#1}}%
    {\newtarget{\expandafter\labelhyp{#1}}{{#2}^{(\hyplink{#1})}}}\endgroup
}
\newcommand{\hyplink}[1]{\newlink{\expandafter\labelhyp{#1}}{\text{H\ref{\expandafter\labelhyp{#1}}}}}

\NewDocumentCommand{\goltarget}{mm}{%
  \begingroup\refstepcounter{gol}\textlabel{\expandafter\labelgol{#1}}%
    {\newtarget{\expandafter\labelgol{#1}}{{#2}^{(\gollink{#1})}}}\endgroup
}
\newcommand{\gollink}[1]{\newlink{\expandafter\labelgol{#1}}{\text{G\ref{\expandafter\labelgol{#1}}}}}

\NewDocumentCommand{\deftarget}{mm}{%
  \begingroup\refstepcounter{def}\textlabel{\expandafter\labeldef{#1}}%
    {\newtarget{\expandafter\labeldef{#1}}{{#2}^{(\deflink{#1})}}}\endgroup
}
\newcommand{\deflink}[1]{\newlink{\expandafter\labeldef{#1}}{\text{G\ref{\expandafter\labeldef{#1}}}}}

\newcommand{\assume}{\hyptarget}
\newcommand{\have}{\hyplink}
\newcommand{\by}{\hyplink}
\newcommand{\wts}{\goltarget}
\newcommand{\suff}{\goltarget}
\newcommand{\shows}{\gollink}
\newcommand{\exact}{\gollink}
\newcommand{\pose}{\deftarget}

\newenvironment{customfont}[2]{%
    \begingroup%
    \fontsize{#1}{#2}\selectfont%
}{%
    \endgroup%
}

\newcommand{\mprlabpos}{lab}
% \newcommand{\mprlabpos}{right}

\let\oldinferrule\inferrule
% adapted from
% https://tex.stackexchange.com/a/340843
\makeatletter
\DeclareDocumentCommand \inferrule { s O {} O{\mprlabpos} m m o}{%
  \IfBooleanTF{#1}%
  {%
    \oldinferrule*[#2]{#4}{#5}%
  }{%
    \oldinferrule*[%
    % HACK WARNING: using #3 and not {#3} so I can add side conditions
    #3=(\IfValueTF{#6}{%
      \begingroup%
      \def\@currentlabelname{\textsc{#2}}%
      \phantomsection%
      \textlabel{#6}{\newtarget{#6}{#2}}%
      \endgroup%
    }%
    {#2}%
    )%
    ]{#4}{#5}%
  }%
}
\makeatother

% surely there is a better way to do this
% this wrapper macro is at least better than the previous hack
\DeclareDocumentCommand \inferruleside { s O {} m m m o}{%
  \IfBooleanTF{#1}{%
    \inferrule*[#2][right={(\({#3}\))}, \mprlabpos]{#4}{#5}[#6]%
  }{%
    \inferrule[#2][right={(\({#3}\))}, \mprlabpos]{#4}{#5}[#6]%
  }%
}

\newcommand{\dash}{\text{-}}
\ebproofnewstyle{back}{proof style=downwards}
\ebproofnewstyle{small}{
  separation = 1em, rule margin = .5ex,
  template = {\small$\inserttext$},
  right label template = {\small\inserttext}
}
\ebproofnewstyle{smaller}{
  separation = 1em, rule margin = .5ex,
  template = {\footnotesize$\inserttext$},
  right label template = {\footnotesize\inserttext}
}

\newenvironment{mathline}{\\[\abovedisplayskip] $}{$ \\[\belowdisplayskip]}

\newcommand{\namerefsc}[1]{\textsc{\nameref{#1}}}

\newcommand{\pinfer}{%
  % \mprset{myfraction=\entailfrac} % breaks on large cases
  \oldinferrule
}

\newcommand{\processnameref}[1]{\nameref{#1}, }
\NewDocumentCommand{\namerefs}{>{\SplitList{,}}om}{%
  \IfValueT{#1}{\ProcessList{#1}{\processnameref} and }\nameref{#2}%
}

%% modified from iris: https://gitlab.mpi-sws.org/iris/iris/-/blob/master/tex/pftools.sty?ref_type=heads
\newcommand{\entailfrac}[2]{%
  \hbox{%
    \ooalign{%
      $\mid\xjoinrel[9]\genfrac{}{}{1.6pt}1{#1}{#2}$\cr%
      $\mid\xjoinrel[9]\color{white}\genfrac{}{}{0.8pt}1{\phantom{#1}}{\phantom{#2}}$%
    }%
  }%
}

% https://tex.stackexchange.com/a/114367
\makeatletter
\providecommand*{\Dashv}{%
  \mathrel{%
    \mathpalette\@Dashv\vDash
  }%
}
\newcommand*{\@Dashv}[2]{%
  \reflectbox{$\m@th#1#2$}%
}
\makeatother

% https://tex.stackexchange.com/a/34716
\makeatletter
\providecommand*{\nest}{%
  \mathrel{%
    \mathpalette\@nest\Rsh
  }%
}
\newcommand*{\@nest}[2]{%
  \raisebox{\depth}{\scalebox{1}[-1]{$\m@th#1#2$}}%
}
\makeatother


\begin{document}

\title{miniDusa: An Extensible Finite-Choice Logic Programming Language (Lightning Talk)}


%% TODO: alphabetical by last name or by first
%% name?? not sure which or it it matters

\author{Ari Prakash}
% \orcid{XXXX-XXXX-XXXX-XXXX}
\affiliation{%
  \institution{Northeastern University}
  \city{Boston}
  \state{MA}
  \country{USA}
}
\email{prakash.ar@northeastern.edu}
\authornote{Both authors contributed equally to this work}


\author{Zachary Eisbach}
% \orcid{0009-0005-3028-7211}
\affiliation{%
  \institution{Northeastern University}
  \city{Boston}
  \state{MA}
  \country{USA}
}
\email{eisbach.z@northeastern.edu}
\authornotemark[1]

% \authorsaddresses{}

\maketitle

\section{Introduction}

\Dusa{}~\cite{martens2025dusa} % maybe cite dusa.rocks impl too?
is a recently designed logic programming language featuring
\emph{mutually exclusive choice} as a primitive 
to admit multiple solutions which
capture a range of possibilities.
% TODO: see if Chris/Rob have any better ideas to characterize
% to enable computation of solutions that satisfy constraints.
% some keywords
% to admit multiple solutions
% exploring the solution space
% characterizing possibility spaces
% ruling out certain combinations

We introduce \miniDusa{}, a \Dusa{}-inspired
% TODO: (nit) italics are awkward here. maybe omit DSL? 
\emph{hosted domain-specific language} (DSL) implemented using Racket and
the \syntaxspec{} metalanguage~\cite{ballantyne2024pearl}. % cite docs?
This architecture lets us provide extensibility and expressivity features
through integration with Racket, essentially ``for free''.
% syntactic extensibility and additional expressive power
% from multi-language interaction with Racket, essentially "for free"".
% inherit tooling and extensibility features from the host

\section{miniDusa by Example}

A finite-choice logic program describes solutions
(i.e. sets of consistent facts), which are obtained using a solver.
To illuminate the semantics of \miniDusa{}, we walk through
a program whose solutions are the 3-colorings of some graph:

% TODO: can we fit in that logic is the boundary with Racket?
\begin{verbatim}
(logic
  (edge 'a 'b) (edge 'a 'c)  ; ... more ...
  ((edge X Y) :- (edge Y X))
  ((node X) :- (edge X _))
\end{verbatim}

A finite-choice logic program consists of facts and rules.
% like Datalog programs. 
The fact \texttt{(edge 'a 'b)} represents an edge from \texttt{'a} to \texttt{'b}.
The premise of the rule \texttt{((edge X Y) :- (edge Y X))} will be
instantiated with this fact to conclude \texttt{(edge 'b 'a)}.
The facts and rules above thus encode an undirected graph.

\begin{verbatim}
  (((color X) is {'r 'g 'b}) :- (node X))
\end{verbatim}
% The conclusion of this rule is a \emph{functional} proposition
% (indicated with \texttt{is}), as opposed to the relational propositions above.
% Each functional proposition has a unique associated value---for instance,
% the value associated to \texttt{color 'a} will be a
% \emph{choice} among \texttt{'r}, \texttt{'g}, and \texttt{'b}.

The \texttt{is} keyword signals a \emph{functional dependency}---%
\texttt{color} acts as a function, mapping each node \texttt{X} to a unique
choice of color. Many solutions will arise,
each corresponding to a distinct combination of choices made.
Without any added constraints, the solutions will enumerate all
possible assignments of colors to nodes. To forbid solutions from
assigning the same color to adjacent nodes, 
the \texttt{forbid} keyword may be used, which
completes the graph coloring example:
% completing the example:
% as follows to complete the \miniDusa{} program:

\begin{verbatim}
  (forbid (edge X Y)
          ((color X) is C)
          ((color Y) is C)))
\end{verbatim}
  % ((ok) is {#t})
  % (((ok) is {#f}) :- (edge X Y)
  %                    ((color X) is C)
  %                    ((color Y) is C)))
% The (functional) fact \texttt{((ok) is {\#t})} will always hold.
% If adjacent nodes have the same color, then \texttt{((ok) is {\#f})}
% will be concluded; this would violate the uniqueness requirement noted
% above, causing any such solutions to be rejected. 

We implement a relatively naïve solver based on the ``fact-set semantics''
described in~\cite{martens2025dusa}. Solving produces a stream of solutions,
each of which may be queried by users.

\section{Abstracting Over Encodings with Macros}
% KEEP IN MIND: this is the part that should be interesting for everyone!
% even if they don't care about logic programs. phrase with that in mind

% TODO: maybe "undirected graph"?
The graph encoding above is cumbersome---a common problem
for encodings of structured data. Unlike in \Dusa{}, users of 
\miniDusa{} can write macros to abstract over encodings and express
constructs in their problem domain in a more natural way.

For instance, the encoding using \texttt{edge} and \texttt{node}
facts and rules can be expressed with an \texttt{undirected-graph}
macro which uses familiar adjacency-list notation:

% TODO: ...s here are a little weird...
\begin{verbatim}
(define-dsl-syntax undirected-graph logic-macro
  (syntax-parser #| ... omitted ... |# ))
(logic
  (undirected-graph edge node
    ('a ['b 'c]) #| ... more ... |# )
  #| ... as before ... |# )
\end{verbatim}

In fact, we (conveniently) implement \texttt{forbid} and similar features
as macros too!
This is an instance of the project architecture simplifying the implementation effort.

The \texttt{define-dsl-syntax} form used above is provided by \syntaxspec{},
which allows \miniDusa{} macros to be carefully expanded into a core syntax
instead of expanding directly to Racket.
This enables a static validation pass before further code generation,
% TODO: do we want a caption on the figure or just a sentence here? maybe space-dependent
as shown in the figure below:

% TODO: not sure why this isn't properly centering...
\begin{figure}[h]
  \begin{center}
    \begin{tikzpicture}[
      box/.style={
        rectangle, draw, align=center,
        minimum width=2.2cm, minimum height=1.1cm,
        },
        label/.style={ midway, font=\small },
        every path/.style={-Stealth, thick}
      ]
      \node[box] (a) at (-2, 2) {\miniDusa{} +\\ extensions};
      \node[box] (b) at ( 2, 2) {Core\\ \miniDusa{}};
      \node[box] (c) at ( 2, 0) {Racket\\with runtime};
      \node[box, shape=ellipse] (d) at (-2, 0) {Solutions};
      
      % janky but it might work
      \draw[loop right] (b) to (b) node[font=\small, right, xshift=1.4cm, yshift=-0.7cm] {Check};
      
      \draw (a) -- (b) node[label, above] {Expand};
      \draw (b) -- (c) node[label, left] {Compile};
      \draw (c) -- (d) node[label, below] {Solve};
    \end{tikzpicture}
  \end{center}
  % \caption{The \miniDusa{} pipeline}
\end{figure}

\section{Host-Language Interoperability}

% syntax spec not only lets us get macros, but also host language interop
% insert: the code with the contrived example
% then, talk about how we can use arbitrary racket functions,
% which is also nice for users: we inherit expressivity and can use libraries/etc

Using \syntaxspec{}, we not only inherit Racket's hygienic macro system---we can
also have richer interoperability with Racket itself. \miniDusa{} allows users
to import pure Racket functions as functional relations, where the
associated value is obtained by calling the function. Ordinary Racket code can
be used to perform otherwise-frustrating computations, perhaps leveraging
existing libraries.

\begin{verbatim}
(define (same-year? y1 y2) #| ... omitted |# )
(logic #:import (same-year?)  ; ... as before
  ((id-dob 'a) is {"01/28/1995"})  ; ... more
  ((edge X Y) :- ((id-dob X) is A)
                 ((id-dob Y) is B)
                 ((same-year? A B) is #t)))
\end{verbatim}

Here, date parsing is handled by Racket to add edges between nodes
with the same birth year. While \Dusa{} has a limited number of
hard-coded built-in functional relations, \miniDusa{} leverages
its status as a hosted DSL to increase expressivity.

% \section{Conclusion}
% TODO: depending on space, include this. not sure what to say to wrap up

\clearpage
\bibliographystyle{ACM-Reference-Format}
\bibliography{refs}

\end{document}
\endinput
